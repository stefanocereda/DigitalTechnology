% !TEX TS-program = xelatex
% !TEX encoding = UTF-8 Unicode


\documentclass[11pt,landscape,a4paper]{article}
\usepackage{multicol}
\usepackage{calc}
\usepackage{ifthen}
\usepackage[landscape]{geometry}
\usepackage{amsmath,amsthm,amsfonts,amssymb}
\usepackage{color,graphicx,overpic}
\usepackage{hyperref}


\usepackage{minted}


% This sets page margins to .5 inch if using letter paper, and to 1cm
% if using A4 paper. (This probably isn't strictly necessary.)
% If using another size paper, use default 1cm margins.
\ifthenelse{\lengthtest { \paperwidth = 11in}}
    { \geometry{top=.5in,left=.5in,right=.5in,bottom=.5in} }
    {\ifthenelse{ \lengthtest{ \paperwidth = 297mm}}
        {\geometry{top=1cm,left=1cm,right=1cm,bottom=1cm} }
        {\geometry{top=1cm,left=1cm,right=1cm,bottom=1cm} }
    }

% Turn off header and footer
\pagestyle{empty}

% Redefine section commands to use less space
\makeatletter
\renewcommand{\section}{\@startsection{section}{1}{0mm}%
                                {-1ex plus -.5ex minus -.2ex}%
                                {0.5ex plus .2ex}%x
                                {\normalfont\large\bfseries}}
\renewcommand{\subsection}{\@startsection{subsection}{2}{0mm}%
                                {-1explus -.5ex minus -.2ex}%
                                {0.5ex plus .2ex}%
                                {\normalfont\normalsize\bfseries}}
\renewcommand{\subsubsection}{\@startsection{subsubsection}{3}{0mm}%
                                {-1ex plus -.5ex minus -.2ex}%
                                {1ex plus .2ex}%
                                {\normalfont\small\bfseries}}
\makeatother

% Define BibTeX command
\def\BibTeX{{\rm B\kern-.05em{\sc i\kern-.025em b}\kern-.08em
    T\kern-.1667em\lower.7ex\hbox{E}\kern-.125emX}}

% Don't print section numbers
\setcounter{secnumdepth}{0}


\setlength{\parindent}{0pt}
\setlength{\parskip}{0pt plus 0.5ex}

%My Environments
\newtheorem{example}[section]{Example}

\DeclareTextFontCommand{\emph}{\bfseries\ttfamily\em}
% -----------------------------------------------------------------------

\begin{document}
\raggedright
\footnotesize
\begin{multicols}{3}


% multicol parameters
% These lengths are set only within the two main columns
%\setlength{\columnseprule}{0.25pt}
\setlength{\premulticols}{1pt}
\setlength{\postmulticols}{1pt}
\setlength{\multicolsep}{1pt}
\setlength{\columnsep}{2pt}

\begin{center}
     \Large{\underline{Python Quick Reference}} \\
\end{center}

\section{Data types}
\emph{boolean} = True/False\\
\emph{integer} = 13\\
\emph{float} = 15.18\\
\emph{string} = ``Hello!''\\
\emph{set} = \{value1, value2, \ldots\}\\
\emph{list} = [value1, value2, \ldots]\\
\emph{dictionary} = \{key1: value1, key2: value2, \ldots\}

\section{Numeric Operators}
\emph{+} addition\\
\emph{-} subtraction\\
\emph{*} multiplication\\
\emph{/} division\\
\emph{\%} modulus\\
\emph{**} exponent

\section{Comparison operators}

\emph{==} equal\\
\emph{!=} different\\
\emph{$>$} higher\\
\emph{$>$=} higher or equal\\
\emph{$<$} lower\\
\emph{$<$=} lower or equal

\section{Boolean operator}
\emph{and} logical AND\\
\emph{or} logical OR\\
\emph{not} logical NOT

\section{String operations}
\emph{string[i]} get character at position i \\
\emph{string[-1]} get last character \\
\emph{string[i:j]} get caracters in range (i, j) \\


\section{List operations}
\emph{mylist = []} create an empty list \\
\emph{mylist[i] = x} save x at index i \\
\emph{mylist[i]} get item with index i \\
\emph{mylist[-1]} get last item \\
\emph{mylist[i:j]} get items from index i to j-1 \\

\section{Dictionary operations}
\emph{mydict = {}} create an empty dictionary \\
\emph{mydict[k] = x} save key k and value x \\
\emph{mydict[k]} retrieve value associated to key k \\
\emph{del mydict[k]} delete the item with key k \\

\section{Set methods}
\emph{myset = set()} create an empty set \\
\emph{myset.add(value)} add value to myset \\
\emph{myset.remove(value)} remove value from myset \\

\section{List methods}
\emph{mylist.append(x)} add x at the end of the list \\
\emph{mylist.remove(x)} remove the first occurence of x \\

\section{Dictionary methods}
\emph{mydict.keys()} get a list of keys \\
\emph{mydict.values()} get a list of values \\
\emph{mydict.items()} get a list of (key, value) pairs \\

\section{Built-in functions}
\emph{print(x)} print x \\
\emph{input(s)} print s and wait for an input \\
\emph{len(x)} number of elements in collection x \\
\emph{min(x)} minimum value in collection x \\
\emph{max(x)} maximum value in collection x \\
\emph{sum(x)} sum of all the items in collection x \\
\emph{range(n1, n2, n3)} sequence of numbers from n1 (included) to n2 (excluded) with steps of n3 \\
\emph{type(x)} type of x (string, float, set, list, \ldots) \\
\emph{str(x)} converts x to string \\
\emph{int(x)} converts x to int \\
\emph{float(x)} converts x to float \\

\section{Conditional statements}
\emph{if statement:}
\begin{minted}{Python}
if <condition>:
    <code>
else if <condition>:
    <code>
...
else:
    <code>
\end{minted}

\emph{Membership testing}
\begin{minted}{Python}
if x in my_collection:
    ...
\end{minted}

\section{Loops}
\begin{minted}{Python}
while <condition>:
    <code>
\end{minted}

\begin{minted}{Python}
for <variable> in <iterable>:
    <code>
\end{minted}

\begin{minted}{Python}
for key in my_dict:
    <code>
\end{minted}

\begin{minted}{Python}
for key, value in my_dict.items():
    <code>
\end{minted}

\emph{break} exit from loop\\
\emph{continue} jump to next iteration

\section{Functions}
\begin{minted}{Python}
def my_func(<params>):
    <code>
    return <data>
\end{minted}

\section{Modules}
\begin{minted}{Python}
import module
module.function()
\end{minted}

\begin{minted}{Python}
from module import function
function()
\end{minted}

% You can even have references
\rule{0.3\linewidth}{0.25pt}
\scriptsize
\bibliographystyle{abstract}
\bibliography{refFile}
\end{multicols}
\end{document}
