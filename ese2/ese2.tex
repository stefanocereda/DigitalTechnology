% !TeX spellcheck = en_GB

\documentclass[aspectratio=169,handout]{beamer}

\usepackage[utf8]{inputenc}
\usepackage[english]{babel}
\usepackage{eurosym}
%\usepackage{appendixnumberbeamer}

\usepackage{tikz}
\usetikzlibrary{shapes,arrows,positioning,calc}

\tikzstyle{startstop} = [ellipse, draw, fill=bg, text=structure, text width=3em, text badly centered, inner sep=0pt]
\tikzstyle{block} = [rectangle, draw, fill=bg, text=structure, text width=5em, text centered]
\tikzstyle{decision} = [diamond, aspect=3, draw, fill=bg, text=structure, text width=3em, text badly centered, inner sep=0pt]
\tikzstyle{input} = [trapezium, draw, fill=bg, text=structure, text width=5em, text centered, trapezium right angle=120]
\tikzstyle{output} = [trapezium, draw, fill=bg, text=structure, text width=5em, text centered, trapezium right angle=120]

\tikzstyle{line} = [draw, -latex']

%\usepackage{xcolor}
\usepackage{listings}

\lstdefinestyle{pseudo}{
	backgroundcolor=\color{bg},
	keywordstyle=\color{alert},
	numberstyle=\tiny\color{fg!50},
	numbers=left,
	%commentstyle=\color{mGreen},
	%stringstyle=\color{mPurple},
    mathescape=true,
	basicstyle=\scriptsize,
    keywords={input,output,begin,end,if,then,else,while,do,for,each,return}
	%tabsize=4,
	%breakatwhitespace=false,
	%breaklines=true,
	%captionpos=b,
	%keepspaces=true,
	%numbersep=5pt,
	%showspaces=false,
	%showstringspaces=false,
	%showtabs=false,
	%language=C
}

\input{../setup-python.tex}

\usetheme[titleformat=smallcaps, numbering=fraction, background=light, progressbar=frametitle]{metropolis}

\title{Digital Technology}
\subtitle{Python Data Types, Control Structures and Functions}

\author{Stefano Cereda\\
		stefano.cereda@polimi.it
	}
\date{31/03/2020}
\institute[PoliMi]{Politecnico Milano}
\logo{\includegraphics[width=15mm]{../logopolimi}}

\setbeamercovered{invisible}

\makeindex

\begin{document}
\begin{frame}
	\maketitle
\end{frame}

\begin{frame}[fragile]{Errata corrige}
    To compute the square root of delta (slide 22) you need to type:
    \begin{minted}[fontsize=\footnotesize]{python}
delta ** (1/2)
    \end{minted}
    and not:
    \begin{minted}[fontsize=\footnotesize]{python}
delta ** 1/2
    \end{minted}
    due to operator precedence
\end{frame}


\begin{frame}[fragile, allowframebreaks]{Sets}
    A \emph{set} object represents an \alert{unordered} collection of \alert{unique} objects.

    They support usual set operations like membership testing, addition and removal.
        \begin{minted}[fontsize=\footnotesize]{python}
myset = set()  # set creation
myset.add(1)  # add an element to a set
myset.add(2)  # add another element
myset.add(1)  # 1 is already in my set, no effect
print(myset)  # will print {1, 2}
print(1 in myset)  # will print True
myset.remove(1)  # remove an element from a set
print(1 in myset)  # will print False
        \end{minted}

        \framebreak
        They also support common sets operations like intersection, union and difference.
        \begin{minted}[fontsize=\footnotesize]{python}
my_set = {2}  # quick way to create a set with element 2
another_set = set()
another_set.add(10)
another_set.add(20)

final_set = myset.union(another_set)  # union of sets {2} and {10, 20}
print(final_set)  # will print {2, 10, 20}
        \end{minted}

        \footnotesize{
        \url{https://docs.python.org/3/tutorial/datastructures.html#sets}
        \url{https://docs.python.org/3.8/library/stdtypes.html#set}
        }
\end{frame}

\begin{frame}[fragile, allowframebreaks]{Lists}
    \emph{Lists} are used to store \alert{ordered sequences} of objects.

        \begin{minted}[fontsize=\footnotesize]{python}
first_list = list()  # empty list creation
first_list.append(1)  # add an element to the end of the list
first_list.append(2)
first_list.append(1)
print(first_list)  # will print [1,2,1]

second_list = [10,20,10,20,10]  # another way to create lists
print(second_list)  # will print [10,20,10,20,10]
print(second_list[0])  # list indexing, will print 10
a = second_list[0:3]  # slice from index 0 (included) to index 3 (excluded)
print(a)  # will print [10, 20, 10]
        \end{minted}

        \framebreak
        \begin{minted}[fontsize=\footnotesize]{python}
print(20 in first_list)  # membership testing as in sets
print(first_list + second_list)  # sum of lists is their concatenation
print(len(first_list))  # number of elements: 3
print(min(first_list))  # 1
print(max(second_list)) # 20
print(sorted(second_list))  # [10, 10, 10, 20, 20]

second_list.append('hello')  # a list can contain heterogeneous types
second_list.append([1,2,3])  # a list can be the element of another list
print(second_list)  # [10, 20, 10, 20, 10, 'hello', [1, 2, 3]]
        \end{minted}

        \footnotesize{
            \url{https://docs.python.org/3/tutorial/datastructures.html#more-on-lists}
            \url{https://docs.python.org/3.8/library/stdtypes.html#list}
        }
\end{frame}

\begin{frame}[fragile]{Dictionaries}
    A \emph{dictionary} is an associative data type.
    Differently from lists, dictionary items are indexed by \alert{keys} instead of numeric indexes.

    A dictionary can be viewed as a set of \alert{key: value} pairs, with the requirement that the keys are unique.

        \begin{minted}[fontsize=\footnotesize]{python}
address = {'alice': 1234, 'bob': 0101}  # dictionary creation
print(address['alice'])  # 1234
address['charlie'] = 5555  # store a new element
del address['bob']  # remove an element
print('charlie' in address)  # True
print('bob' not in address)  # True
print(1234 in address)  # False, we check on keys
        \end{minted}

        \footnotesize{
    \url{https://docs.python.org/3/tutorial/datastructures.html#dictionaries}
    \url{https://docs.python.org/3.8/library/stdtypes.html#mapping-types-dict}
}
\end{frame}

\begin{frame}[fragile]{While}
    The \emph{while} statement is used to repeat the execution of a suite of statements as long as an expression is
    true:

    \begin{minipage}{0.49\textwidth}
        \begin{minted}[fontsize=\footnotesize]{python}
i = 0
while i < 10:
    print(i)
    i = i + 1
        \end{minted}
    \end{minipage}
    \begin{minipage}{0.49\textwidth}
        \resizebox{!}{4cm}{
            \begin{tikzpicture}[auto]
                \node [startstop] (start) {Begin};
                \node [block, below = of start] (init) {i = 0};
                \node [decision, below = of init] (while) {i < 10?};
                \node [output, right = of while, text width=1cm] (print) {i};
                \node [block, right = of print] (increment) {i=i+1};
                \node [startstop, below = of while] (end) {End};

                \path [line] (start) -- (init);
                \path [line] (init) -- (while);
                \path [line] (while) -- node [near start]{True} (print);
                \path [line] (print) -- (increment);
                \path [line] (increment) -- ($ (increment) + (0, 1) $) -| (while);
                \path [line] (while) -- node [near start] {False} (end);
            \end{tikzpicture}
        }
    \end{minipage}

    This code prints all the numbers from 0 (included) to 10 (excluded)
\end{frame}

\begin{frame}[fragile]{For}
    The \emph{for} statement is used to iterate over the elements of an iterable object, like strings, lists, sets or
    dictionaries.

    \begin{minted}[fontsize=\footnotesize]{python}
my_list = [1, 10, 100, 1000]
for number in my_list:
    print(number)
    \end{minted}
\end{frame}

%\begin{frame}{Break and continue}
%    The \emph{break} statement is used to exit prematurely from an iteration.
%
%    The \emph{continue} statement is used to advance to the next iteration.
%\end{frame}

\begin{frame}[fragile]{Example: factorial}
    Write a Python program to compute the factorial of a number N.
    \pause

    \begin{minted}[fontsize=\footnotesize]{python}
number = int(input("Give me a number: "))
result = 1
factor = 1
while factor <= number:
    result = result * factor
    factor = factor + 1
print("{}! = {}".format(number, result))
    \end{minted}
\end{frame}

\begin{frame}[fragile]{Example: factorial with range}
    We can also use the \emph{range(a, b)} function to obtain an \emph{iterator} over the numbers from a (included) to b
    (excluded).
    \begin{minted}[fontsize=\footnotesize]{python}
number = int(input("Give me a number: "))
result = 1
for factor in range(1, number+1):
    result = result * factor
print("{}! = {}".format(number, result))
    \end{minted}
\end{frame}

\begin{frame}[fragile]{Example: computation of $\pi$}
    Write a python program to compute an approximation of $\pi$.
    Ask the number of iterations to the user and print an approximation every 10 iterations.

    \pause
    \begin{minted}[fontsize=\footnotesize]{python}
max_iters = int(input("How many iterations? "))
pi = 0
# range(max) === range(0, max)
# range(max_iters) will go from 0 to max_iters-1 (included)
for n_iter in range(max_iters):
    if n_iter % 2 == 0:
        pi = pi + 1 / (2*n_iter + 1)
    else:
        pi = pi - 1 / (2*n_iter + 1)
    # we are counting iterations from zero
    if (n_iter+1) % 10 == 0:
        print("The value of pi after {} iterations is {}".format(n_iter+1, 4*pi))
print("The final approximation is {}".format(4*pi))
    \end{minted}
\end{frame}

\begin{frame}[fragile]{List comprehension}
    List comprehensions provide a concise way to create lists:

    \begin{minipage}{0.49\textwidth}
    \begin{minted}[fontsize=\footnotesize]{python}
squares = []
for x in range(10):
    squares.append(x**2)
print(squares)
    \end{minted}
    \end{minipage}
    \begin{minipage}{0.49\textwidth}
    \begin{minted}[fontsize=\footnotesize]{python}
squares = [x**2 for x in range(10)]
print(squares)
    \end{minted}
    \end{minipage}
\end{frame}

\begin{frame}[fragile]{Dictionary comprehension}
    We also have dictionary comprehensions:

    \begin{minipage}{0.49\textwidth}
    \begin{minted}[fontsize=\footnotesize]{python}
squares = {}
for x in range(10):
    squares[x] = x**2
print(squares)
    \end{minted}
    \end{minipage}
    \begin{minipage}{0.49\textwidth}
    \begin{minted}[fontsize=\footnotesize]{python}
squares = {x: x**2 for x in range(10)}
print(squares)
    \end{minted}
    \end{minipage}
\end{frame}

\begin{frame}{Exercise: exam}
    Write a python program that:
    \begin{enumerate}
        \item reads a sequence of student IDs, terminated by -1
        \item for each student, reads the exam result
        \item prints the IDs of the students who passed the exam
        \item prints the minimum, maximum and average mark
        \item prints the standard deviation of the marks
    \end{enumerate}
\end{frame}

\begin{frame}{Exercise: compound interest}
    An amount of \euro5000 is deposited into a savings account at an annual interest rate of 5\%.

    Compute and show the amount of money at the end of each year, for 10 years.
\end{frame}

\begin{frame}{Exercise: uppercase}
    Write a python program that asks a the user to insert a phrase.
    The program reads the phrase and considers the first word, repeating it in upper case.

    You can convert a string in upper case with the \emph{.upper()} method.
\end{frame}

\begin{frame}[fragile]{Python function}
    We have seen some \emph{functions} (len(), min(), max(), set(), \ldots).

    A function is a piece of code that receives some \emph{parameters} and produces a result.

    \begin{minipage}{0.49\textwidth}
        Function definition:
        \begin{minted}[fontsize=\footnotesize]{python}
def my_function(a, b):
    value = a + b
    return value
        \end{minted}
    \end{minipage}
    \begin{minipage}{0.49\textwidth}
        Function calling:
        \begin{minted}[fontsize=\footnotesize]{python}
print(my_function(3,2))
        \end{minted}
    \end{minipage}
\end{frame}

\begin{frame}[fragile]{Function example: average}
    You should write functions to execute common tasks, like computing the average value of an iterable object:

        \begin{minted}[fontsize=\footnotesize]{python}
def average(iterable):
    return sum(iterable) / len(iterable)
        \end{minted}

        You can now compute the average of a list:
        \begin{minted}[fontsize=\footnotesize]{python}
print(average([1,2,3]))
        \end{minted}
\end{frame}

\begin{frame}[fragile]{Importing modules}
    You can write many functions in a file (called \emph{module}) and use them in another script by \emph{importing}
    them.

    If you save the average function in a file called myfunctions.py, you can use it in another script with:
    \begin{minted}[fontsize=\footnotesize]{python}
import myfunctions
print(myfunctions.average([1,2,3]))
    \end{minted}

    Python comes with a standard library, offering a wide variety of facilities:
    \url{https://docs.python.org/3/library/}

    For instance, we have the \emph{math} module for the \emph{sqrt} function:
    \begin{minted}[fontsize=\footnotesize]{python}
import math
print(math.sqrt(9))
    \end{minted}

    We will add other modules to python with \emph{pip} and \url{https://pypi.org/}.
\end{frame}

\begin{frame}{Question time}
    You need to create a data structure so to be able to recover the company name from its market abbreviation.
    (E.g., you know AAPL and need to retrieve Apple).
    Which is the best choice?
    \begin{enumerate}
            \item a list
            \item a set
            \item a dictionary
    \end{enumerate}
\end{frame}

\begin{frame}{Question time}
    What would you use to store all the names of the companies traded in the nasdaq index?
    \begin{enumerate}
            \item a list
            \item a set
            \item a dictionary
    \end{enumerate}
\end{frame}

\begin{frame}{Question time}
    You want to store all the closing values obtained by a certain company in the past week (one value per day), what
    would you use?
    \begin{enumerate}
            \item a list
            \item a set
            \item a dictionary
    \end{enumerate}
\end{frame}

\begin{frame}[fragile]{Question time}
    What is the value of my\_list?
    \begin{minted}[fontsize=\footnotesize]{python}
my_list = [number % 2 for number in range(10, 12)]
    \end{minted}
    \begin{enumerate}
        \item  10, 11, 12
        \item  0, 1, 0
        \item  0, 1
        \item  1, 0, 1
    \end{enumerate}
\end{frame}

\end{document}
