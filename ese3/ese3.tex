% !TeX spellcheck = en_GB

\documentclass[aspectratio=169,]{beamer}

\usepackage[utf8]{inputenc}
\usepackage[english]{babel}
%\usepackage{appendixnumberbeamer}

\usepackage{tikz}
\usetikzlibrary{shapes,arrows,positioning,calc}

\tikzstyle{startstop} = [ellipse, draw, fill=bg, text=structure, text width=3em, text badly centered, inner sep=0pt]
\tikzstyle{block} = [rectangle, draw, fill=bg, text=structure, text width=5em, text centered]
\tikzstyle{decision} = [diamond, aspect=3, draw, fill=bg, text=structure, text width=3em, text badly centered, inner sep=0pt]
\tikzstyle{input} = [trapezium, draw, fill=bg, text=structure, text width=5em, text centered, trapezium right angle=120]
\tikzstyle{output} = [trapezium, draw, fill=bg, text=structure, text width=5em, text centered, trapezium right angle=120]

\tikzstyle{line} = [draw, -latex']

%\usepackage{xcolor}
\usepackage{listings}

\lstdefinestyle{pseudo}{
	backgroundcolor=\color{bg},
	keywordstyle=\color{alert},
	numberstyle=\tiny\color{fg!50},
	numbers=left,
	%commentstyle=\color{mGreen},
	%stringstyle=\color{mPurple},
    mathescape=true,
	basicstyle=\scriptsize,
    keywords={input,output,begin,end,if,then,else,while,do,for,each,return}
	%tabsize=4,
	%breakatwhitespace=false,
	%breaklines=true,
	%captionpos=b,
	%keepspaces=true,
	%numbersep=5pt,
	%showspaces=false,
	%showstringspaces=false,
	%showtabs=false,
	%language=C
}

\input{../setup-python.tex}

\usetheme[titleformat=smallcaps, numbering=fraction, background=light, progressbar=frametitle]{metropolis}

\title{Digital Technology}
\subtitle{Json and Pandas}

\author{Stefano Cereda\\
stefano.cereda@polimi.it
}
\date{21/04/2020}
\institute[PoliMi]{Politecnico Milano}
\logo{\includegraphics[width=15mm]{../logopolimi}}

\setbeamercovered{invisible}

\makeindex

\begin{document}
\begin{frame}
    \maketitle
\end{frame}

\begin{frame}{Json}
    You have seen at lesson that JSON is built with name: value pairs.
    A JSON file can thus be described with a Python dictionary.

    The JSON module allows to convert a between python objects and their textual JSON representation.
\end{frame}

\begin{frame}[fragile]{json.dumps}
    The \emph{dumps} function converts python variables to strings:
    \begin{minted}[autogobble]{Python}
        import json
        my_dict = {'firstName': 'John',
                   'age': 25
                  }
        print(type(my_dict))

        js_repr = json.dumps(my_dict)
        print(js_repr)
        print(type(js_repr))
    \end{minted}
\end{frame}

\begin{frame}[fragile]{json.loads}
    The \emph{loads} function is used to convert from string to python objects:
    \begin{minted}[autogobble]{Python}
        # continues from previous slide
        my_obj = json.loads(js_repr)
        print(my_obj)
        print(type(my_obj))
    \end{minted}
\end{frame}


\begin{frame}[fragile]{json.load and json.dump}
    Usually we want to read and write JSON \emph{files} instead of dealing with strings.

    The \emph{dump} and \emph{load} allow to work with files (no trailing \emph{s} as they do not work with
    \emph{s}trings.

    To use them, we first need to be able to open files in Python:
    \begin{minted}[autogobble, fontsize=\tiny]{Python}
        import json
        my_dict = {'firstName': 'John', 'age': 25 }

        # Open a file in write mode
        with open('./my_file.json', 'w') as outfile:
            # outfile is a variable pointing to my_file.json, opened in write mode
            json.dump(my_dict, outfile)
        # outfile is closed when we exit the with statement

        # now we open the file in read mode and read its content:
        with open('./my_file.json', 'r') as infile:
            loaded_obj = json.load(infile)

        if loaded_obj != my_dict:
            print("JSON does not work!")
    \end{minted}
\end{frame}

\begin{frame}[fragile]{The open function}
    The \emph{open} function opens a file and return the corresponding file object.
    Its two most important parameters are the \emph{filename}, representing the path to the file to be opened, and the
    \emph{mode}: a string representing how we should open the file ('w' for write, 'r' for read).

    \url{https://docs.python.org/3/library/functions.html#open}
\end{frame}

\begin{frame}[fragile]{The with statement}
    When we finish processing the file we should close it:
    \begin{minted}[autogobble]{Python}
        outfile = open('./my_file.json', 'w')
        json.dump(my_dict, outfile)
        outfile.close()
    \end{minted}
    However, doing this we must remember to close the file, even if the write instruction fails.

    The with statement is a much more convenient way to deal with file objects, as we have seen in previous examples.
\end{frame}



\begin{frame}[fragile]{Installing external modules with pip}
    We have seen how to import a module from the Python standard library:
    \begin{minted}[autogobble]{Python}
        import math
        from random import uniform
        from json import dumps
    \end{minted}

    However, there exists many modules which are not included in the standard library (numpy, pandas, scipy,
    scikit-learn, etc\ldots)

    You can install, upgrade, and remove packages using a program called \emph{pip}:
    \begin{minted}[autogobble]{text}
        pip install numpy
        pip install pandas
        pip uninstall numpy
        pip freeze
    \end{minted}
\end{frame}

\begin{frame}[fragile]{Virtual environments}
    Different python applications will require different versions of the packages.

    The solution is to create a virtual environment, a directory that contains a
    Python installation plus a number of additional packages.

    Virtualenv creation:
    \begin{minted}[autogobble]{bash}
        python -m venv my_venv
        # equivalent: virtualenv my_venv
    \end{minted}

    And activation:
    \begin{minted}[autogobble]{text}
        my_venv\Scripts\activate.bat  # Windows
        source my_venv/bin/activate   # Linux and MacOS
    \end{minted}

    Using \emph{pip} with an active virtual environment will work on the virtual environment.
\end{frame}

\begin{frame}{PyCharm, pipenv and anaconda}
    In PyCharm every project has its own virtual environment, and you can easily manage its packages
    (File/Settings/Project/Project Interpreter).

    Other widely used programs for managing virtual environments are pipenv and anaconda.
\end{frame}

\begin{frame}{Pandas}
    \emph{Pandas} is a data analysis and manipulation library.

    Pandas allows you to open tabular data (like csv, spreadsheet and databases) and obtain Python objects called
    \emph{DataFrame}.

    \begin{figure}
        \centering
        \def\svgwidth{\textwidth}
        \input{pandas.pdf_tex}
    \end{figure}
\end{frame}

\begin{frame}[fragile]{DataFrame creation and column selection}
    Let's manually create a DataFrame:
    \begin{minted}[autogobble]{Python}
        import pandas as pd
        my_df = pd.DataFrame({
            "Name": ["Alice", "Bob", "Charlie"],
            "Age": [25, 30, 35]
        })
        print(my_df)
    \end{minted}

    \pause
    Each column is a pandas \emph{Series}.
    You can select a column using square brackets:
    \begin{minted}[autogobble]{Python}
        print(my_df['Age'])
    \end{minted}
\end{frame}

\begin{frame}[fragile]{Simple analysis}
    We can use the usual min, max, sum, \ldots functions on Series:
    \begin{minted}[autogobble]{Python}
        print(sum(my_df['Age']) / len(my_df['Age']))
    \end{minted}

    Pandas also provide the \emph{.describe()} method, which computes some statistics (in form of another DataFrame):
    \begin{minted}[autogobble]{Python}
        print(my_df.describe())
        print(type(my_df.describe()))
    \end{minted}
\end{frame}

\begin{frame}[fragile]{Opening CSV files}
    We can easily open a csv file with the \emph{read\_csv} function:
    \begin{minted}[autogobble]{Python}
        import pandas as pd
        df = pd.read_csv('aapl.us.txt')
        print(df.head(5))  # will print the first 5 rows (tail for last)
        print(df.describe())
        print(df.dtypes)  # data types of each column
        print(df.info())  # technical information
    \end{minted}

    You can find the csv file on beep.
\end{frame}

\begin{frame}[fragile]{Saving DataFrames}
    You can save a DataFrame to any of the format supported by Pandas:
    \begin{minted}[autogobble]{Python}
        import pandas as pd
        df = pd.read_csv('aapl.us.txt')
        df.to_excel('./aapl.xslsx')
    \end{minted}
\end{frame}

\begin{frame}[fragile]{Filtering specific rows}
    We are interested in the days with closing price above 22:
    \begin{minted}[autogobble]{Python}
        import pandas as pd
        df = pd.read_csv('aapl.us.txt')
        above_22 = df[df['Close'] > 22]
        print(above_22)
    \end{minted}

    We are interested in the days with closing price above 22 and below 30:
    \begin{minted}[autogobble]{Python}
        middle = df[(df['Close'] > 22) & (df['Close'] < 30)]
        print(above_22)
    \end{minted}

    Notice the single \&
\end{frame}

\begin{frame}[fragile]{Filtering specific rows and columns}
    We want to obtain the opening prices of the days with closing price above 22:
    \begin{minted}[autogobble]{Python}
        prices = df[df['Close'] > 22]['Open']
        prices = df.loc[df['Close'] > 22, 'Open']
    \end{minted}

    In the first version we obtain a new DataFrame with the interesting rows and all the columns and then we select the
    'Open' column.

    In the second row we directly select the interesting rows and columns on the original DataFrame.
\end{frame}

\begin{frame}[fragile]{loc vs iloc}
    The \emph{.loc} operator is used when dealing with boolean indexing (as we did before), the \emph{.iloc} operator is
    used for integer indexing (you know the row and column numbers).

    We want to obtain the opening prices of the rows 2 to 10:
    \begin{minted}[autogobble]{Python}
        print(df.columns)  # 'Open' is the second
        prices = df.iloc[1:10, 1]
    \end{minted}
\end{frame}

\begin{frame}[fragile]{Modifying data}
    We want to write 1 as the closing price for all the days with opening price above 2.

    We have seen two ways to select the interesting data, let's use them to modify the values:
    \begin{minted}[autogobble]{Python}
        print(df.describe())
        prices_1 = df[df['Open'] > 2]['Close'] = 1
        prices_2 = df.loc[df['Open'] > 2, 'Close'] = 1
        print(prices_1.describe())
        print(prices_2.describe())
    \end{minted}

    \pause
    The version without the .loc works on a view of the original DataFrame, so you will lose any change you make.
\end{frame}
\end{document}
